\documentclass{article}
\usepackage{indentfirst}
\begin{document}

\title{EmpathicEngine Development Diary}
\author{Zhuang Chang}
\date{\today}

\maketitle


\tableofcontents
\section{Backlog}
\paragraph{Scientific Problems}

Definition of agent, software agent, hardware agent

Definition of empathy:

Seeing with the eyes of another, listening with the ears the another and feeling with 
the heart of another.


Theory of emotion and affective computing





\begin{enumerate}
\item[1] relationship between human and agent in collaborative MR system
\begin{enumerate}
    \item[1.1]whether setting collaborative task in virtual reality can improve empathy? 
    
    Teacher demonstrates knowledge to students is a kind of collaboration.
    
    Two people collaborate on keeping a ball on a rod is another type of collaboration task that 
    might help improve understanding of each other.

    \item[1.2] when and how could agent help improve understanding between two collaborators?

    Human plays the role of observer while the agent takes the role of target

    \item[1.3] How does social realtionship influence empathy between collaborators?
 
\end{enumerate} 

\item[2]pros and cons of integrating an agent into a collborative system?
\begin{enumerate}
    \item[2.1]Whether an agent with the capacity of empathy in MR collobrative system can help 
    enhance humanbeings' ability of empathy? Whether agent modulating collaboration inforamtion actively can help imporve understanding?
    \item[2.2]Maybe in other situations it inhibits collaboration and might become intrusive.
 
\end{enumerate}

\item[3]How can agent generate empathic representation properly?
\begin{enumerate}
    \item[3.1]Emotion theory
    \item[3.2]Data processing methods 
    \item[3.2]evaluation criteria or methods
 
\end{enumerate} 

\item[4]How to calculate empathy quantitively?


\end{enumerate}

\paragraph{Empathy}
De Waal\cite{preston2002empathy} defines empathy as the capacity to a) be affected by share the emotional
 state of the another, b) assess the reasons for the others state, and 
 c) identify with the other, adopting his or her perspectives.


\paragraph{Agent}
An agent can make decisions automatically according to its own knowledge
\paragraph{Emotion theory}

\paragraph{Affective Computing}

\paragraph{Collaborative Mixed reality}




In MR based collaborative environment, we research empathy between or among 
follolwing objects:
\begin{enumerate}
\item[1] human-human
\item[2] human-Agent
\item[3] agent-human
\item[4] agent-Agent
\item[5] human-agent-human
\item[6] agent-human-human
\item[7] agent-human group
\item[8] human-agent group
\item[9] huamn group-agent group     
\end{enumerate}

\section{PhD Research Plan}
\subsection{Summary}
\paragraph{Preface}
Construct an engine to support most mixed
reality (MR) related scientific research.

Agent in MR environment help enhance understand and 
feeling of social presence.

Agent can manipulate (augment or depress) elements from the computer side to 
change focus to create new meaning of current job.

\paragraph{MR content}
3D model will be placed on the web server.

\paragraph{Interaction}
Interaction depends intensively on terminal devices

\paragraph{Display}
Display is also constrained by terminal devices.

\paragraph{AI}
Integrate artificial intelligence to construct Agent
that can sense, express and regulate emotion

\subparagraph{Datasource}
Data comprises physical and physiological signals.
\begin{enumerate}
\item[1)]
physical signals includes images, voices and so on.
\item[2)]
Physiological signals comprises Electroencephlogram(EEG), Electrocardiogram
(ECG), etc..
\end{enumerate}

\subsection{Experiment}
\begin{enumerate}
\item[1)]
human adjust their physical or physiological signal to reach a level and computer
will change volume or picture color
\item[2)]
Adjust physical and physiological state to colloboratively finish a task and
feel oneself's state to infer other's state. This can be used for psycological 
trainning where a healthy people can help psychological disabled one to 
feel the normal inner state. 
\item[3)] 
One adjust his or her physical or physiological level to change a VR scene by which 
the observer's state can be influenced. 
\end{enumerate}

\section{DevDiary}
\section{20200822}
\begin{enumerate}
    \item[1)] 
    Finish mingw download and configuration
    \item[2)] 
    Download mysql and configure it with the program
    \item[3)] 
     
\end{enumerate}
\section{20201123}
ThinkPhp is kind of framework for database Interaction facilitated at server side. For user side Html, CSS and Js are some common techniques for constructing web pages connected to database.

create an agent that can recognize virtual humanbeing's face emotion in VR

webpage can be helpful for database management, thus a database is necessary first.

methodology is design hunman computer interactive systems and use behaviour statistics mathmetical methods to validate the usability.

behaviour of statistics is important for this way and SPSS will be helpful for data analysing.


\bibliographystyle{unsrt}
\bibliography{References}
\end{document}